% StarPU --- Runtime system for heterogeneous multicore architectures.
%
% Copyright (C) 2018-2020  Université de Bordeaux, CNRS (LaBRI UMR 5800), Inria
%
% StarPU is free software; you can redistribute it and/or modify
% it under the terms of the GNU Lesser General Public License as published by
% the Free Software Foundation; either version 2.1 of the License, or (at
% your option) any later version.
%
% StarPU is distributed in the hope that it will be useful, but
% WITHOUT ANY WARRANTY; without even the implied warranty of
% MERCHANTABILITY or FITNESS FOR A PARTICULAR PURPOSE.
%
% See the GNU Lesser General Public License in COPYING.LGPL for more details.
%
\input{./version.sty}
\setlength{\parskip}{0pt}
\begin{titlepage}
\vspace*{4cm}
{\Huge \textbf{StarPU Internal Handbook}}\\
\rule{\textwidth}{1.5mm}
\begin{flushright}
{\Large for StarPU \STARPUVERSION}
\end{flushright}
\rule{\textwidth}{1mm}
~\\
\vspace*{15cm}
\begin{flushright}
Generated by Doxygen.
\end{flushright}
\end{titlepage}

\begin{figure}[p]
This manual documents the internal usage of StarPU version \STARPUVERSION. Its contents
was last updated on \STARPUUPDATED.\\

Copyright © 2009–2019 Université de Bordeaux

Copyright © 2010-2019 CNRS

Copyright © 2011-2019 Inria

\medskip

\begin{quote}
Permission is granted to copy, distribute and/or modify this document
under the terms of the GNU Free Documentation License, Version 1.3 or
any later version published by the Free Software Foundation; with no
Invariant Sections, no Front-Cover Texts, and no Back-Cover Texts. A
copy of the license is included in the section entitled “GNU Free
Documentation License”.
\end{quote}
\end{figure}

\pagenumbering{roman}
\setcounter{tocdepth}{2}
\tableofcontents
\pagenumbering{arabic}
\hypersetup{pageanchor=true,citecolor=blue}

\chapter{Introduction}
\label{index}
\hypertarget{index}{}
\input{index}

\chapter{Star\+PU Core}
\label{StarPUCore}
\hypertarget{StarPUCore}{}
\input{StarPUCore}

\chapter{Module Index}
% StarPU --- Runtime system for heterogeneous multicore architectures.
%
% Copyright (C) 2021-2024   University of Bordeaux, CNRS (LaBRI UMR 5800), Inria
%
% StarPU is free software; you can redistribute it and/or modify
% it under the terms of the GNU Lesser General Public License as published by
% the Free Software Foundation; either version 2.1 of the License, or (at
% your option) any later version.
%
% StarPU is distributed in the hope that it will be useful, but
% WITHOUT ANY WARRANTY; without even the implied warranty of
% MERCHANTABILITY or FITNESS FOR A PARTICULAR PURPOSE.
%
% See the GNU Lesser General Public License in COPYING.LGPL for more details.
%
\section{Modules}
Here is a list of all modules\+:\begin{DoxyCompactList}
\item \contentsline{section}{Workers}{\pageref{group__workers}}{}
\end{DoxyCompactList}


\chapter{Module Documentation}
\label{ModuleDocumentation}
\hypertarget{ModuleDocumentation}{}

\input{group__workers}

\chapter{Index}
\printindex

\end{document}
