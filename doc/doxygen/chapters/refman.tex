\documentclass{book}
\usepackage[a4paper,top=2.5cm,bottom=2.5cm,left=2.5cm,right=2.5cm]{geometry}
\usepackage{makeidx}
\usepackage{natbib}
\usepackage{graphicx}
\usepackage{multicol}
\usepackage{float}
\usepackage{listings}
\usepackage{color}
\usepackage{ifthen}
\usepackage[table]{xcolor}
\usepackage{textcomp}
\usepackage{alltt}
\usepackage{ifpdf}
\ifpdf
\usepackage[pdftex,
            pagebackref=true,
            colorlinks=true,
            linkcolor=blue,
            unicode
           ]{hyperref}
\else
\usepackage[ps2pdf,
            pagebackref=true,
            colorlinks=true,
            linkcolor=blue,
            unicode
           ]{hyperref}
\usepackage{pspicture}
\fi
\usepackage[utf8]{inputenc}
\usepackage{mathptmx}
\usepackage[scaled=.90]{helvet}
\usepackage{courier}
\usepackage{sectsty}
\usepackage{amssymb}
\usepackage[titles]{tocloft}
\usepackage{doxygen}
\lstset{language=C++,inputencoding=utf8,basicstyle=\footnotesize,breaklines=true,breakatwhitespace=true,tabsize=8,numbers=left }
\makeindex
\setcounter{tocdepth}{3}
\renewcommand{\footrulewidth}{0.4pt}
\renewcommand{\familydefault}{\sfdefault}
\hfuzz=15pt
\setlength{\emergencystretch}{15pt}
\hbadness=750
\tolerance=750
\begin{document}
\hypersetup{pageanchor=false,citecolor=blue}
\begin{titlepage}
\vspace*{4cm}
{\Huge \textbf{StarPU Handbook}}\\
\rule{\textwidth}{1.5mm}
\begin{flushright}
{\Large for StarPU 1.2.0}
\end{flushright}
\rule{\textwidth}{1mm}
~\\
\vspace*{15cm}
\begin{flushright}
Generated by Doxygen $doxygenversion on $datetime
\end{flushright}
\end{titlepage}

\begin{figure}[p]
This manual documents the usage of StarPU version 1.2.0. Its contents
was last updated on 24 May 2013.\\

Copyright © 2009–2013 Université de Bordeaux 1\\

Copyright © 2010–2013 Centre National de la Recherche Scientifique\\

Copyright © 2011, 2012 Institut National de Recherche en Informatique et Automatique\\

\medskip

\begin{quote}
Permission is granted to copy, distribute and/or modify this document
under the terms of the GNU Free Documentation License, Version 1.3 or
any later version published by the Free Software Foundation; with no
Invariant Sections, no Front-Cover Texts, and no Back-Cover Texts. A
copy of the license is included in the section entitled “GNU Free
Documentation License”.
\end{quote}
\end{figure}

\clearemptydoublepage
\pagenumbering{roman}
\tableofcontents
\clearemptydoublepage
\pagenumbering{arabic}
\hypersetup{pageanchor=true,citecolor=blue}

\part{Using StarPU}

\chapter{Introduction}
\label{index}
\hypertarget{index}{}
\input{index}

\chapter{Building and Installing Star\-P\-U}
\label{buildingAndInstalling}
\hypertarget{buildingAndInstalling}{}
\input{buildingAndInstalling}

\chapter{Basic Examples}
\label{basicExamples}
\hypertarget{basicExamples}{}
\input{basicExamples}

\chapter{Advanced Examples}
\label{advancedExamples}
\hypertarget{advancedExamples}{}
\input{advancedExamples}

\chapter{How to optimize performance with StarPU}
\label{optimizePerformance}
\hypertarget{optimizePerformance}{}
\input{optimizePerformance}

\chapter{Performance Feedback}
\label{performanceFeedback}
\hypertarget{performanceFeedback}{}
\input{performanceFeedback}

\chapter{Tips and Tricks to know about}
\label{tipsTricks}
\hypertarget{tipsTricks}{}
\input{tipsTricks}

\chapter{StarPU MPI Support}
\label{mpiSupport}
\hypertarget{mpiSupport}{}
\input{mpiSupport}

\chapter{StarPU FFT Support}
\label{fftSupport}
\hypertarget{fftSupport}{}
\input{fftSupport}

\chapter{C Extensions}
\label{cExtensions}
\hypertarget{cExtensions}{}
\input{cExtensions}

\chapter{SOCL OpenCL Extensions}
\label{soclOpenclExtensions}
\hypertarget{soclOpenclExtensions}{}
\input{soclOpenclExtensions}

\chapter{Scheduling Contexts in StarPU}
\label{schedulingContexts}
\hypertarget{schedulingContexts}{}
\input{schedulingContexts}

\chapter{Scheduling Context Hypervisor}
\label{schedulingContextHypervisor}
\hypertarget{schedulingContextHypervisor}{}
\input{schedulingContextHypervisor}

\part{Inside StarPU}

\chapter{Deprecated List}
\label{deprecated}
\hypertarget{deprecated}{}
\input{deprecated}

\chapter{Module Index}
% StarPU --- Runtime system for heterogeneous multicore architectures.
%
% Copyright (C) 2021-2024   University of Bordeaux, CNRS (LaBRI UMR 5800), Inria
%
% StarPU is free software; you can redistribute it and/or modify
% it under the terms of the GNU Lesser General Public License as published by
% the Free Software Foundation; either version 2.1 of the License, or (at
% your option) any later version.
%
% StarPU is distributed in the hope that it will be useful, but
% WITHOUT ANY WARRANTY; without even the implied warranty of
% MERCHANTABILITY or FITNESS FOR A PARTICULAR PURPOSE.
%
% See the GNU Lesser General Public License in COPYING.LGPL for more details.
%
\section{Modules}
Here is a list of all modules\+:\begin{DoxyCompactList}
\item \contentsline{section}{Workers}{\pageref{group__workers}}{}
\end{DoxyCompactList}

\chapter{Data Structure Index}
\input{annotated}

\chapter{Module Documentation a.k.a StarPU's API}

\input{group__Versioning}
\input{group__Initialization__and__Termination}
\input{group__Standard__Memory__Library}
\input{group__Workers__Properties}
\input{group__Data__Management}
\input{group__Data__Interfaces}
\input{group__Data__Partition}
\input{group__Multiformat__Data__Interface}
\input{group__Codelet__And__Tasks}
\input{group__Insert__Task}
\input{group__Explicit__Dependencies}
\input{group__Implicit__Data__Dependencies}
\input{group__Performance__Model}
\input{group__Profiling}
\input{group__Theoretical__lower__bound__on__execution__time}
\input{group__CUDA__Extensions}
\input{group__OpenCL__Extensions}

%\input{group__MIC__Extensions}
%\input{group__SCC__Extensions}

\input{group__Miscellaneous__helpers}
\input{group__FxT__Support}
\input{group__FFT__Support}
\input{group__MPI__Support}
\input{group__Task__Bundles}
\input{group__Task__Lists}
\input{group__Parallel__Tasks}
\input{group__Running__Drivers}
\input{group__Expert__Mode}
\input{group__StarPU-Top__Interface}

\input{group__Scheduling__Contexts}
\input{group__Scheduling__Policy}
\input{group__Scheduling__Context__Hypervisor}

\printindex
\end{document}

\chapter{}
\label{}
\hypertarget{}{}
\input{}
